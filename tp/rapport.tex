% Created 2021-06-29 mar. 08:54
% Intended LaTeX compiler: pdflatex
\documentclass[11pt]{article}
\usepackage[utf8]{inputenc}
\usepackage[T1]{fontenc}
\usepackage{graphicx}
\usepackage{grffile}
\usepackage{longtable}
\usepackage{wrapfig}
\usepackage{rotating}
\usepackage[normalem]{ulem}
\usepackage{amsmath}
\usepackage{textcomp}
\usepackage{amssymb}
\usepackage{capt-of}
\usepackage{hyperref}
\usepackage{color}
\usepackage[natbibapa]{apacite}
\usepackage[french]{babel}
\usepackage{a4wide}
\usepackage{mathpazo}
\usepackage{titlesec}
\titlelabel{\thetitle.\quad}
\usepackage[usenames,dvipsnames]{xcolor} % For colors with friendly names
\usepackage{float}
\usepackage{url}
%% For DOI hyperlinks in biblio:
\usepackage{doi}
\renewcommand{\doiprefix}{}
\author{Frédéric Santos\thanks{frederic.santos@u-bordeaux.fr}}
\date{\today}
\title{Utiliser GNU Emacs et Org mode dans le milieu académique}
\hypersetup{
 pdfauthor={Frédéric Santos},
 pdftitle={Utiliser GNU Emacs et Org mode dans le milieu académique},
 pdfkeywords={},
 pdfsubject={},
 pdfcreator={Emacs 27.2 (Org mode 9.4.6)}, 
 pdflang={French}}
\begin{document}

\maketitle
\tableofcontents


\section{GNU Emacs}
\label{sec:orgd335271}
\subsection{Historique}
\label{sec:orgf1d1849}
Emacs est une \emph{famille} d'éditeurs de texte avancés, dont le représentant le plus populaire est GNU Emacs. À l'origine, \og Emacs\fg{} signifie \og Editing MACroS\fg{}, et le système de macros est encore une des forces d'Emacs aujourd'hui.

Emacs a été créé en 1976 par Richard Stallman, en repartant d'un éditeur de texte propriétaire de l'époque. GNU Emacs en a offert une version libre et extensible.

\subsection{Philosophie}
\label{sec:orgea11323}
Emacs est un éditeur de texte :
\begin{itemize}
\item extensible (possibilité d'installer de nombreux plug-ins, ou \emph{packages}) ;
\item personnalisable ;
\item documenté (une aide exhaustive est intégrée).
\end{itemize}

\subsection{Communauté}
\label{sec:org7738f39}
Les débutants peuvent trouver de l'aide notamment via :
\begin{itemize}
\item le site \href{https://emacs.stackexchange.com/}{StackExchange} ;
\item un channel Telegram intégré grâce à \href{https://zevlg.github.io/telega.el/}{Telega}.
\end{itemize}

\subsection{Quelques concurrents}
\label{sec:org0b1262e}
La table \ref{table-concurrents} présente quelques autres éditeurs de texte avancés.

\begin{table}[htbp]
\centering
\begin{tabular}{lll}
\hline
Éditeur & OS & Commentaire\\
\hline
Atom & Tous & Très populaire sous Windows\\
nano & Linux, Macs OS & Un éditeur plus léger\\
vi / vim & Linux & L'autre éditeur \og expert\fg{}\\
\hline
\end{tabular}
\caption{Quelques concurrents d'Emacs. \label{table-concurrents}}

\end{table}

\subsection{Identité visuelle}
\label{sec:orgdb5f751}
Le logo d'Emacs est présenté en figure \ref{fig-logo}.

\begin{figure}[htbp]
\centering
\includegraphics[width=0.2 \textwidth]{./images/emacs.png}
\caption{Le logo d'Emacs. \label{fig-logo}}
\end{figure}

\section{Org mode}
\label{sec:org1fc5c47}
\subsection{Historique}
\label{sec:org9f15e0a}
Org mode est un mode majeur d'Emacs, créé en 2003 par l'astronome Carsten Dominik. Il s'agissait à l'origine d'un système avancé de prise de notes pour scientifiques,mais il a rapidement évolué pour devenir un système très polyvalent d'organisation au quotidien et de rédaction de documents académiques.

\subsection{Polyvalence}
\label{sec:org636aeea}
Nous aurons vu au cours de cette formation deux fonctionnalités principales d'Org mode :
\begin{enumerate}
\item Le langage de balisage Org pour rédiger des documents
\item La gestion de tâches pour s'organiser au quotidien
\end{enumerate}

\section{Des exemples d'utilisation d'Org mode dans le milieu académique}
\label{sec:orgfb5c6ac}
Org mode est très adapté à l'écriture d'articles de recherche reproductibles \citep{schulte2012_MultiLanguageComputingEnvironment}. On trouve aisément des exemples d'articles rédigés à 100\% en Org mode, certains étant assez anciens \citep[e.g.,][]{dye2011_ModelbasedAgeEstimate}, d'autres très récents \citep{santos2020_ModernMethodsOld,fraga2021_MultipleSimultaneousSolution}.

\bibliography{biblio}
\bibliographystyle{apacite}
\end{document}